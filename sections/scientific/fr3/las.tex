\subsubsection{FR03: Investigation of the architecture of an attention-based sequence to sequence model.} % corrected 99

\paragraph{Introduction}

In this section, the \textit{Listen, Attend and Spell} (LAS) model will be investigated and presented based on its paper\cite{chan2015listen}. LAS is a model which takes acoustic features as inputs and outputs characters as outputs. We define, $x \ = \ (x_1,...,x_T)$ an input sequence containing Mel-filterbank features and $y \ = \ (\langle sos\rangle,y_1,...,y_S,\langle eos\rangle)$ an ouput sequence of characters where $y_i \in \{a,b,...,z,0,...,9,\langle \ \rangle,\langle ,\rangle,\langle .\rangle,\langle '\rangle, \langle unk\rangle\}$. The tokens $\langle sos\rangle$ and $\langle eos\rangle$ represent the start and end of a sentence respectively. The LAS model represents a conditional distribution over previous characters $y_j$ where $j < i$ and the input signal $x$ for each output character $y_i$. Using the chaing rule, we get:

\begin{equation}
	P(y|x) = \prod_{i} P(y_i|x,y_j)
\end{equation}

This model is composed of two modules: the listener and the speller. The listener $Listen$ represents an acoustic model encoder, while the speller $AttendAndSpell$ is an attention-based character decoder. The first module converts the input signal $x$ into a higher-level feature representation $h$, whereas the speller module processes $h$ as input and computes a probability distribution over a sequence of characters. A representation of the LAS model with these two modules is given in Fig. \ref{las_model}.

\begin{align}
	h &= Listen(x)\\
	P(y|x) &= AttendAndSpell(h,y)
\end{align}

\begin{figure}[h]
  \centering
  \includegraphics[scale=0.18]{las_architecture.jpg}
	\caption{Representation of Listen, Attend and Spell (LAS) model.\cite{las_model} The listener, a pyramidal BLSTM, encodes the input signal $x$ to a higher level feature representation $h$, while the speller decodes the output $y$ from this high level representation $h$.}
  \label{las_model}
\end{figure}

\paragraph{Sequence to sequence learning} % corrected 84

Despite deep neural networks being successfully used in various classification applications which categorise fixed input vectors to output classes, to process variable-length sequences, neural networks are combined with sequential models such as Hidden Markov Models (HMMS). These combined models result in a statistical machine translation instead of an end-to-end neural machine translation. The disadvantage is that statistical models are difficult to be trained end-to-end.\cite{chan2015listen} The encoder-decoder architecture copes with this drawback. Models based on this architecture can process input and output sequences of variable lengths. This architecture design is built on two main components. The first is called the encoder which transforms a variable-length sequence into a fixed shape representation. The other component is the decoder and it maps this fixed representation to a variable-length sequence.\cite{zhang2020dive} An illustration of this architecture is given in Fig. \ref{encoder_decocer}.

\begin{figure}[H]
  \centering
  \begin{tikzpicture}
		\node (r0) at (-3,0) [draw,minimum height=0.5cm] {Input};
		\node (r1) at (-1.5,0) [draw,minimum height=1cm] {Encoder};
		\node (r2) at (0,0) [draw,minimum height=0.5cm] {State};
		\node (r3) at (1.5,0) [draw,minimum height=1cm] {Decoder};
		\node (r4) at (3,0) [draw,minimum height=0.5cm] {Output};
    
		\draw[stateTransition] (r0) -- (r1);
		\draw[stateTransition] (r1) -- (r2);
		\draw[stateTransition] (r2) -- (r3);
		\draw[stateTransition] (r3) -- (r4);

  \end{tikzpicture}
	\caption{The encoder-decocer architecture.}
  \label{encoder_decocer}
\end{figure}


Sequence to sequence learning is a process based on the encoder-decoder architecture. During training, the model provides the decoder with the labels, while during inference, the model executes a beam search to obtain a relevant output for later predictions.

These models are improved when combined with an attention mechanism. It supplies the decoder more information when producing the output. This mechanism propagates information more effectively from the encoder to the decoder at each time step.

This learning framework has been applied to many domains, such as machine translation, image captioning, etc. This learning process is not limited to a particular domain and it could be applied to build ASR systems.

\paragraph{Listen} % corrected 86

The $Listen$ operation is defined by a Bidirectional LSTM RNN (BLSTM) which is structured into a pyramid. This pyramidal structure reduces the length of input $x$ from $T$ to the length $U$ of $h$. Applying the BLSTM directly to the $Listen$ operation will converge slowly and performs poorly since the $AttendAndSpell$ operation has difficulties to extract useful information from larger input sequences.\cite{chan2015listen} To mitigate this issue, a pyramid BLSTM (pBLSTM) is used. At each stacked pBLSTM layer, the time resolution of the input is decreased by a factor of 2. In normal deep BLSTM RNNs, the output at the $i$-th time step and $j$-th layer is computed as follows:

\begin{equation}
	h_i^j = BLSTM(h_{i-1}^j, h_i^{j-1})
\end{equation}

Compared to the BLSTM architecture, the pBLSTM model concatenates the outputs of consecutive steps of each layer and passes it to the next layer:

\begin{equation}
	h_i^j = pBLSTM(h_{i-1}^j, \lbrack h_{2i}^{j-1}, h_{2i+1}^{j-1}\rbrack)
\end{equation}

The LAS model stacks 3 pBLSTMs above the BLSTM input layer. This decreases the time resolution $2^3 = 8$ times, allowing the attention model to derive important information from an input sequence with fewer time steps. An illustration of the $Listen$ operation is given in Fig. \ref{las_model}

\paragraph{Attend and Spell} % corrected 99

The $AttendAndSpell$ operation uses an attention-based LSTM transducer to compute at every output step a probability distribution over the next character based on all the characters seen before. The context vector $c_i$ is generated by an attentation mechanism. The decoder state $s_i$ is computed from the previous state $s_{i-1}$, the previously produced character $y_{i-1}$ and the context $c_{i-1}$. Further, the probability distribution $y_i$ is a function of the decoder state $s_i$ and the context $c_i$. Respectively,

\begin{align}
	c_i &= AttentionContext(s_i,h)\\
	s_i &= RNN(s_{i-1}, y_{i-1}, c_{i-1})\\
	P(y_i|x, y_{<i}) &= CharacterDistribution(s_i,c_i)
\end{align}

where $CharacterDistribution$ represents an MLP which character outputs are activated by a softmax function and $RNN$ is equal to a 2 layer LSTM network.\\

At each timestep $i$ of the decoder, the $AttentionContext$ function generates for each timestep $u$ a scalar energy $e_{i,u}$ from $h_u \in h$ and $s_i$. The scalar energy $e_{i,u}$ is transformed into a probability distribution over timesteps applying a softmax function. This distribution is used to create the context vector $c_i$ when linearly blending it with the listener features $h_u$ at different timesteps:

\begin{align}
	e_{i,u} &= \langle \phi(s_i), \psi(h_u)\rangle\\
	\alpha_{i,u} &= \frac{exp(e_{i,u})}{\sum_u exp(e_{i,u})}\\
	c_i &= \sum_u \alpha_{i,u}h_u
\end{align}

where $\phi$ and $\psi$ are MLP networks. $c_i$ can be interpreted as a collection of weighted features of h.

\paragraph{Learning}

\begin{equation}
	\max_{\theta} \sum_i log P (y_i|x, y_{<i}^*;\theta)
\end{equation}

\begin{align}
	\tilde{y_i} \sim CharacterDistribution(s_i,c_i) \\
	\max_\theta \sum_i log P(y_i|x,\tilde{y}_{<i};\theta)
\end{align}


\paragraph{Decoding and Rescoring}

\begin{equation}
	\hat{y} = arg \max_{y} log P(y|x)
\end{equation}


\begin{equation}
	s(y|x) = \frac{log P(y|x)}{|y|_c}+\lambda log P_{LM}(y)
\end{equation}
