% corrected LN 100

\subsubsection{FR02: Mel Filterbank feature extraction in Python}~\label{torchaudio}

We use the Pytorch module \textit{Torchaudio}\cite{torchaudio} to extract the Mel Filterbanks from the dataset. Instead of processing the whole dataset, the following steps are applied to one single audio file $f$:\\

\begin{enumerate}[label=\arabic*.]
  \item Let $f$ be an audio file sampled at $16kHz$, \textit{Torchaudio} loads the song as follows:
\begin{lstlisting}
import torchaudio
import torchaudio.compliance.kaldi.fbank as fbank

waveform, sr = torchaudio.load(filepath)
\end{lstlisting}
with $waveform$ being the audio time series and $sr$ the sample rate of $waveform$.\\

% \item The next step is to pad the time series $y$ if it is smaller than $1s$:
% \begin{lstlisting}
% import numpy as np

% if y.size < 16000:
%   rest = 16000 - y.size
%   left = rest // 2
%   right = rest - left
%   y = np.pad(y, (left, right), 'reflect'))
% \end{lstlisting}
% The time series is reflected on both sides to get the correct size.\\

\item The Mel Filterbanks can be extracted with:
\begin{lstlisting}
filterbank = fbank(waveform, num_mel_bins=40, channel=-1, sample_frequency=sample_rate)
\end{lstlisting}
The function \textit{fbank()} returns a \textit{torch.Tensor} with the dimension:
\begin{equation*}
  shape=(padded\_window\_size // 2 + 1, \ m) 
\end{equation*}
with $padded\_window\_size // 2 + 1$ being the number of frames and $m$ the number of Mel-filters.
\item The last step is to flatten this matrix to a vector which can be stored in a data frame for later training.
\begin{lstlisting}
row = filterbank.transpose(0, 1).unsqueeze(0)
\end{lstlisting}
After this operation, the matrix $filterbank$ is transposed and flattened. This concatenates the columns of $filterbank$ together forming a vector.
\end{enumerate}

% These steps are then applied to the entire dataset. Each computed vector is appended to the data frame which is going to be stored as a $.csv$ file. The entire code snippet is given in the Appendix in Figure~\ref{mfccsnip}.
