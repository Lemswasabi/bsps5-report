% corrected VD 81

\subsubsection{FR02: Deep Learning and Artificial Neural Networks}\label{ann}~\\

The notions of Deep Learning and examples of ANNs presented in this section are
based on the textbook \textit{Deep Learning}\cite{Goodfellow-et-al-2016}. These
presentations should only give a high-level introduction to these notions.

\subsubsection{Deep Learning}~\\

% rephrase dots

Artificial Intelligence (AI) is a field with many practical applications, one of
them being the task to understand speech. AI deals with challenging problems
which are easy to perform but hard to describe formally by humans. Deep Learning
is a solution to these intuitive problems. This approach to AI allows computer
systems to study from experience and understand the world through a hierarchical
stack of concepts. The computing system gathers knowledge from experience which
eliminates the need to describe formal rules. The computer understands complex
concepts with this stack of concepts by decomposing them with simpler ones. The
representation of this hierarchy of concepts shows a deep graph with many
layers, hence the name Deep Learning.~\cite{Goodfellow-et-al-2016}
