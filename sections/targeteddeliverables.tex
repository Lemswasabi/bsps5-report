% corrected VD 97

\subsection{Targeted Deliverables}

\subsubsection{Scientific deliverables}

The scientific aspect covered by this Bachelor Semester Project is the
architecture of attention-based sequence to sequence models. The focus of the
scientific presentation will be on the architecture of this model and how it
compares to other state-of-the-art models. Further, the model's performance
will be assessed when using different hyperparameters. One of the main
deliverables is to present how to extract the features from the dataset with
Filter Banks. 

% One of the main deliverables is to present how to extract the features from
% the dataset with Mel-frequency cepstral coefficients (MFFCs).  Additionally,
% we present the notions of Deep learning. This paper gives a brief
% introduction to Artificial Neural Networks (ANN) and their application for
% classification.  Further, we extend this scientific presentation by diving
% deeper into two different ANNs; Feedforward Neural Networks and Recurrent
% Neural Networks (RNN).  Special cases of RNN such as gated RNNs are
% furthermore investigated. This includes long short-term memory (LSTM) and
% gated recurrent units (GRU). Finally, we compare their performances obtained
% after being trained on the given audio dataset.

\subsubsection{Technical deliverables} 

The technological aspect of this project is the reuse and adaptation of an
existing implementation of the attention-based speech recognition model.
Additionally, the LibriSpeech dataset will be preprocessed and used for the
model’s training. The goal is to deploy its training on an HPC environment for
efficient computing.

% The other main deliverable for this paper is the implementation of an end-to-end
% speech recognition model based on the four Neural Networks, i.e. feed-forward
% neural networks, standard RNNs, LSTM and GRU. Additionally, we collect a small
% dataset of Luxembourgish spoken words $w \in \{'0',\dots,'9'\}$ to train our
% model to recognize these words. To the best of our knowledge, there are no
% publically available datasets for speech recognition of the Luxembourgish
% language.

